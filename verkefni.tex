% TODO make sure at the end that 'Caratheodory' is nowhere
% TODO fix open intervals. they look ugly
% TODO skrifað Weierstrass rétt
% TODO decide if 0 is in \mathbb{N} and make sure it's consistent
% TODO read and mention Globevnik
% in F. and M. Riesz chapter ralk about why the function spaces are seperable
\documentclass[a4paper,12pt,twoside,BCOR=10mm]{scrbook}

% Packages
\usepackage{ucs}
\usepackage[utf8x]{inputenc}
\usepackage[icelandic, english]{babel}
\usepackage{t1enc}
\usepackage{graphicx}
\usepackage[intoc]{nomencl}
\usepackage{enumerate,color}
\usepackage{url}
\usepackage[pdfborder={0 0 0}]{hyperref}
\usepackage{appendix}
\usepackage{eso-pic}
\usepackage{amsmath}
\usepackage{amsthm}
\usepackage{amssymb}
\usepackage[nottoc]{tocbibind}
\usepackage[sort&compress,authoryear]{natbib}
\usepackage[sf,normalsize]{subfigure}
\usepackage[format=plain,labelformat=simple,labelsep=colon]{caption}
\usepackage{placeins}
\usepackage{tabularx}
\usepackage{mathabx}
\usepackage{mathtools}
% Configurations
\graphicspath{{figs/}}

\newtheorem{theorem}{Theorem}
\newtheorem{corollary}{Corollary}
\newtheorem{remark}{Remark}
\newtheorem{example}{Example}
\newtheorem{lemma}{Lemma}
\newtheorem{definition}{Definition}
\renewcommand{\Re}{\text{Re }}
\renewcommand{\Im}{\text{Im }}

\setlength{\parskip}{\baselineskip}
\setlength{\parindent}{0cm}
\raggedbottom
% \setkomafont{subsection}{\normalfont\sffamily}

% Eins og templatið á að vera
% \setkomafont{captionlabel}{\itshape}
% \setkomafont{caption}{\itshape}

% Mun fallegri lausn
\setkomafont{captionlabel}{\itshape}
\setkomafont{caption}{\itshape}
\setkomafont{section}{\FloatBarrier\Large}
\setcapwidth[l]{\textwidth}
\setcapindent{1em}


% Times new roman
%\usepackage[T1]{fontenc}
%\usepackage{mathptmx}

%%%%%%%%%%% MODIFY THESE LINES ONLY %%%%%%%%%%%%%%%%%%%%%%%%%%%%%%%%%%%%%%%%%%%%%%%%%%%%%%%%%
\def\thesisyear{2020}       						% Year thesis submitted
\def\thesismonth{XXmonth}						% Month thesis submitted
\def\thesisauthor{Bergur Snorrason}					% Thesis authoreiningaraðferðinni
\def\thesistitle{XXTitle} 						% Title of thesis
\def\thesisshorttitle{XXShort title (50 characters including spaces)} 	% Title of thesis
\def\thesiscredits{XX} 							% Credits awarded for the project
\def\thesissubject{XX}
\def\thesiskind{M.Sc.}							% Masters of PhD thesis
\def\thesiskindformal{Magister Scientiarum}				% Masters of PhD thesis
\def\thesisnroftutors{1}						% Number of tutors
\def\thesisschool{School of Engineering and {Natural Sciences}}		% School
\def\thesisfaculty{XX}							% Faculty
\def\thesisaddress{XXFaculty street address}				% Office address
\def\thesispostalcode{XXFaculty postal code, Reykjavik}			% Office address
\def\thesistelephone{525 4000}						% Office telephone
\def\thesispublisher{XX}						% Publisher
\def\thesistutors{XXNN1 \\ XXNN2}
\def\thesisrepresentative{XXNN3}					% Tutors name
\def\thesiscommittee{XXNN4 \\ XXNN5 }
\def\thesiskeywords{Keyword1, Keyword2, Keyword3}			% Keywords
\def\thesisISBN{XX}           						% Thesis ISBN number
\def\thesisdedication{Dedication}
\def\thesisPrinting{Háskólaprent, Fálkagata 2, 107 Reykjavík}

% Function to add footer to frontpage
\newcommand\BackgroundPic{
\put(0,0){
\parbox[b][\paperheight]{\paperwidth}{
\vfill
\centering
\hspace*{-0.6cm}
\includegraphics[width=\paperwidth,height=\paperheight,
keepaspectratio]{foot}
}}
\setlength{\unitlength}{\paperwidth}
\begin{picture}(0,0)(0,-0.15)
\put(0,0){\color{white}\parbox{1\paperwidth}{\centering\bfseries\sffamily \Large Faculty of \thesisfaculty \\
University of Iceland\\
\thesisyear}}
\end{picture}
}

\begin{document}

\begin{titlepage}
\thispagestyle{empty}
\AddToShipoutPicture*{\BackgroundPic}
%
\begin{center}
\vspace*{1cm}
\includegraphics[width=43.6mm]{ui_1_cmyk}\\
\vspace*{3.0cm}
\huge \sffamily \bfseries \thesistitle

\vspace*{5.5cm}
\normalfont \Large \sffamily \thesisauthor
\AddToShipoutPicture*{\BackgroundPic}
\vfill

\end{center}

\newpage 
\thispagestyle{empty} \mbox{}
\newpage
\vspace*{1.35cm}
\thispagestyle{empty}
\begin{center}

\Large \textbf{\sffamily{\MakeUppercase{\thesistitle}}} \\

\vspace*{1.7cm}
\sffamily{\thesisauthor} \\
\vspace*{1.8cm}
\normalsize \thesiscredits~ECTS thesis submitted in partial fulfillment of a \\
\textit{\thesiskindformal} degree in \thesissubject

\vspace*{1.0cm}
\large
\ifnum\thesisnroftutors >1 Advisors \\ \thesistutors \\ \vspace*{0.4cm}
\else Advisor \\ \thesistutors \\ \vspace*{1.04cm}
\fi
Faculty Representative \\
\thesisrepresentative

\vspace*{0.4cm}
M.Sc. committee \\
\thesiscommittee

\vfill
Faculty of \thesisfaculty \\
\thesisschool \\
University of Iceland \\
Reykjavik, \thesismonth~\thesisyear
\newpage
\end{center}
 \newpage
 \thispagestyle{empty}
 \mbox{} \vfill
 % \setcounter{page}{0} \renewcommand{\baselinestretch}{1.5}\normalsize
 \sffamily{\thesistitle} \\
 \sffamily{\thesisshorttitle} \\
 \thesiscredits ~ECTS thesis submitted in partial fulfillment of a \thesiskind~degree in \thesissubject
\\ \\
Copyright \textcopyright~\thesisyear~ \thesisauthor \\
All rights reserved \\


Faculty of \thesisfaculty \\
\thesisschool \\
University of Iceland \\
\thesisaddress \\
\thesispostalcode, Reykjavik \\
Iceland

Telephone: \thesistelephone \\ \\
\vspace*{\lineskip}

Bibliographic information: \\
\thesisauthor, \thesisyear, \thesistitle, \thesiskind~thesis, Faculty of \thesisfaculty, University of Iceland. \\

ISBN~\thesisISBN

Printing: \thesisPrinting \\
Reykjavik, Iceland, \thesismonth~\thesisyear \\
\newpage
\thispagestyle{empty} \mbox{}
\vfill
\begin{center}\textit{\thesisdedication}\end{center} \vspace*{5cm}
\vfill 

\thispagestyle{empty}
\cleardoublepage
\end{titlepage}


% \dedication{\textit{Dedication} \small \\ Tileinkun má sleppa og skal þá fjarlægja blaðsíðuna. \\
% Tileinkun skal birtast á oddatölu blaðsíðu (hægri síðu).}
\pagenumbering{roman}

\setcounter{page}{5}
\section*{\huge Abstract}
Útdráttur á ensku sem er að hámarki 250 orð.
\vfill \vspace*{1cm}
\section*{\huge Útdráttur}
Hér kemur útdráttur á íslensku sem er að hámarki 250 orð. Reynið að koma útdráttum á eina blaðsíðu en ef tvær blaðsíður eru nauðsynlegar á seinni blaðsíða útdráttar að hefjast á oddatölusíðu (hægri síðu).
\vfill
\newpage

\chapter*{Preface}
Formála má sleppa og skal þá fjarlægja þessa blaðsíðu. Formáli skal hefjast á oddatölu blaðsíðu og nota skal Section Break (Odd Page).

Ekki birtist blaðsíðutal á þessum fyrstu síðum ritgerðarinnar en blaðsíðurnar teljast með og hafa áhrif á blaðsíðutal sem birtist með rómverskum tölum fyrst á efnisyfirliti.

\tableofcontents
\listoffigures
\listoftables



\chapter*{Abbreviations}
\addcontentsline{toc}{chapter}{Abbreviations}
Í þessum kafla mega koma fram listar yfir skammstafanir og/eða breytuheiti. Gefið kaflanum nafn við hæfi, t.d. Skammstafanir eða Breytuheiti. Þessum kafla má sleppa ef hans er ekki þörf. \\
The section could be titled: Glossary, Variable Names, etc.



\chapter*{Acknowledgments}
\addcontentsline{toc}{chapter}{Acknowledgments}
Í þessum kafla koma fram þakkir til þeirra sem hafa styrkt rannsóknina með fjárframlögum, aðstöðu eða vinnu. T.d. styrktarsjóðir, fyrirtæki, leiðbeinendur, og aðrir aðilar sem hafa á einhvern hátt aðstoðað við gerð verkefnisins, þ.m.t. vinir og fjölskylda ef við á. Þakkir byrja á oddatölusíðu (hægri síðu).

\pagenumbering{arabic}
\setcounter{page}{1}
\chapter{Introduction}
% TODO define 'extends'


\chapter{Preliminaries}
\section{Measure theory}
\section{The disk algebra}
\section{Complex analysis in one variable}
\subsection{Carathéodory}
\begin{definition}
A continuous function $\gamma: [0, 1] \rightarrow \mathbb{C}$ is said to be a \emph{Jordan curve} if $\gamma(0) = \gamma(1)$ and
\[
	\gamma(s) = \gamma(t) \implies s = t \tag*{for all $s, t \in ]0, 1[$.}
\]
\end{definition}
The definition above can be restated as: A Jordan curve is a closed simple curve.
The term `simple' here means that the curve is not self-intersecting.
The name stems from a famous result by Camille Jordan stating that $\mathbb{C} \setminus \gamma([0, 1])$ has two connected component, one of which is simply connected.
The simply connected component will be called \emph{the domain bounded by $\gamma$}.
The proof of this result is rather technical and outside the scope of this thesis \cite{fdsa}\cite{fdas}. % TODO add citation to whyburn chapter 3 and green-krantz p. 393
The result is however necessary to make statement such as `let $U$ be the domain bounded by $\gamma$'.
An example of this is the following theorem:
\begin{theorem}[Carathéodory]
Let $\Omega_1$ and $\Omega_2$ be domains in $\mathbb{C}$ each bounded by a Jordan curve and $\Phi: \Omega_1 \rightarrow \Omega_2$ be a conformal mapping.
There exists a continuous injection $\hat{\Phi}: \overline{\Omega_1} \rightarrow \overline{\Omega_2}$ that extends $\Phi$.
\end{theorem}
A proof of this can be found in section $13.2$ of \cite{green and krantz}. % TODO motivate the theorem a bit
\subsection{The Riemann Mapping Theorem} 
We will start of with a definition. % TODO reword
\begin{definition}
A map $f: U \rightarrow V$, with open $U, V \subset \mathbb{C}$ is said to be \emph{conformal} if it is holomorphic, bijective, and its inverse is holomorphic.
\end{definition}
\begin{remark}
The fact that the inverse is homomorphic is actually redundant.
It can be shown \cite{green and krantz bls. 165} that if $h$ is holomorphic and $h'$ vanishes to order $k$ at $z_0$ then $h$ is $(k + 1)$-to-one in a neighbourhood of $z_0$.
So if $h$ is bijective then $h'$ vanishes nowhere.
\end{remark}
\begin{theorem}[Riemann mapping theorem]
If $U \subset \mathbb{C}$, $U \neq \mathbb{C}$ is homeomorphic to $\mathbb{D}$ then there exists a conformal mapping from $\mathbb{D}$ to $U$.
\end{theorem}
\begin{proof}
A proof of this is rather involved and can be found in \cite{proof of Riemann mapping theorem}. % TODO add citation
\end{proof}
\begin{corollary}
If $U$ and $V$ are both homeomorphic to $\mathbb{D}$ then there exists a conformal mapping from $U$ to $V$.
\end{corollary}
\begin{proof}
The Riemann mapping theorem gives us $\Phi_1$, a conformal mapping from $\mathbb{D}$ to $U$, and $\Phi_2$, a conformal mapping from $\mathbb{D}$ to $V$.
The desired conformal mapping from $U$ to $V$ is then $\Phi_2 \circ \Phi_1^{-1}$.
\end{proof}
In this thesis the disk algebra $\mathcal{A}$ is of special interest so a version of the Riemann mapping theorem that considers continuity at the boundary is desirable.
We can combine the Riemann mapping theorem and Carathéodory's theorem to achieve the desired theorem.
The only thing to prove is that a homeomorphism $f$ maps the boundary of a bounded set $U$ to the boundary of $f(U)$ and that a Jordan curve under a homeomorphism is still a Jordan curve.
\begin{corollary}
If $K$ is homeomorphic to $\overline{\mathbb{D}}$ then there exists a continuous, injective $\Phi: \overline{\mathbb{D}} \rightarrow K$ such that its restriction to $\mathbb{D}$ is a conformal mapping.
\end{corollary}
\begin{proof}
Firstly, let $U$ be bounded, open set in $\mathbb{C}$,
	$f: \overline{U} \rightarrow \mathbb{C}$ be an injective continuous map,
	and let's show that $\partial f(U) = f(\partial U)$.
Let $p \in \partial U$ and $B_r(z) = \{z \in \mathbb{C};\ |r - z| < r\}$.
Let's also assume that $f(p) \in f(U)$ and show that it leads to a contradiction.
There exists an $r > 0$ such that $B_r(p) \subset f(U)$, because $U$ is open and therefore $f(U)$ as well.
This gives us as an open neighbourhood $f^{-1}(B_r(p)) \subset U$ of $p$, but $p \in \partial U$ implies that on such neighbourhood exists.
So $f(p)$ is not in $f(U)$ but it is in $\overline{f(U)}$, implying $\partial f(U) \supset f(\partial U)$.
We can use the same argument to show that $\partial f(U) \subset f(\partial U)$, since $f$ is bijective if we consider it as a map from $\overline{U}$ into $f(\overline{U})$.
So $f(\partial U) = \partial f(U)$. % TODO skoða þennan hluta mun betur!!

Secondly, let
	$U$ and $V$ be open sets in $\mathbb{C}$,
	$\gamma: [0, 1] \rightarrow \partial \mathbb{D}$ Jordan curve,
	$f: U \rightarrow V$ be homeomorphism,
Let $\lambda = f \circ \gamma$ and $s, t \in ]0, 1[$ such that $\lambda(s) = \lambda(t)$.
It suffices to show that $s = t$, since $\lambda(0) = \lambda(1)$ obviously holds.
We have that $f(z) = f(w)$ implies $x = w$, since $f$ is bijective, and therefore injective.
So $\lambda(s) = (f \circ \gamma)(s) = (f \circ \gamma)(t) = \lambda(t)$ implies that $\gamma(s) = \gamma(t)$.
But $\gamma$ is a Jordan curve, so $\gamma(s) = \gamma(t)$ implies $s = t$.
So $\lambda$ is also a Jordan curve. % TODO reread and make sure this still makes sense

Finally, we can prove the corollary.
Let $f$ be a homeomorphism from $\overline{\mathbb{D}}$ to $K$.
The Riemann mapping theorem also gives us a conformal map $\Phi: \overline{\mathbb{D}} \rightarrow K$.
We know that $\overline{\mathbb{D}}$ is compact, so $K = f(\overline{\mathbb{D}})$ is also compact, since the image of a compact set under a continuous mapping is also compact.
So $K$ is also bounded.
So $f(\partial \mathbb{D} = \partial f(\mathbb{D}) = \partial K$ according to our first step and $\partial K$ is a Jordan curve according to the second step, since $\partial \mathbb{D}$ is a Jordan curve.
So $\Phi$ is a conformal map between two domains bounded, each bounded by a Jordan curve.
This allows us to use Carathéodory's theorem to extend $\Phi$ continuously and injectively to $\overline{\mathbb{D}}$, concluding the proof.
\end{proof}
The Riemann mapping theorem is a strong tool when analyzing holomorphic functions on simply connected domains.
We can often solve things for the unit disk (or unit square as in \ref{Rudin-Carleson reference here}) and then map that solution to a general simply connected domain. % TODO add ref
\subsection{Weierstrass M-test}
\section{Functional analysis}
\subsection{Hahn-Banach}
\subsection{Riesz representation theorem}
\section{Miscellaneous}
% TODO Ef summa fágaðra falla er samleitini í j.m. þá er hún fáguð
% TODO Ef summa samfelldra falla er samleitini í j.m. þá er hún samfelld
% TODO Neðra mat þríhyrningsójöfnunnar, ||x| - |y|| <= |x - y|



\chapter{Rudin-Carleson theorem}
\begin{theorem}[Rudin-Carleson theroem]
Let $E$ be a closed subset of $\mathbb{T}$ of Lebesgue-measure $0$, let $f$ be a continuous function on $E$ and let $T$ be a simply connected subset of $\mathbb{C}$ such that $f(\mathbb{\overline{D}}) \subset T$.
Then there exists an $F \in \mathcal{A}$, such that $F = f$ on $E$ and $F(\mathbb{\overline{D}}) \subset T$. % TODO skilgreina allt
\end{theorem}
We will break the proof into several lemmas.  % TODO how many?
\begin{lemma}
Let $H$ be a closed set of Lebesbue-measure $0$.
Then there exists an integrable function $\mu > 1$ such that
	$\mu$ is continuous on $\mathbb{T} \setminus H$,
	$\mu = +\infty$ on $H$,
	if $w \in H$ then $\mu(z) \xrightarrow{z \rightarrow w} +\infty$, and
	$\mu$ has a bounded derivative on any closed subarc of $\mathbb{T} \setminus H$.
\end{lemma}
\begin{proof}
TODO
\end{proof}
\begin{lemma}
If $f$ is a simple continuous function on $E$ such that $\Re f \geq 0$, then there exists an $F \in \mathcal{A}$ such that $F = f$ on $E$ and $\Re F \geq 0$ on $\overline{\mathbb{D}}$.
\end{lemma}
\begin{proof}
It suffices to show that this holds if $f$ takes only two values on $E$, since simple functions are finite linear combinations of characteristic functions.
Let's assume these values are $0$ and $\alpha \neq 0$, with $\Re \alpha \geq 0$, $E_0 = f^{-1}(0)$, and $E_1 = f^{-1}(\alpha)$.
Our assumption that $f$ only takes two values then implies that $E_0 \cap E_1 = E$.

Let $u_H(z)$ be the Poisson integral of the function from the above lemma with $H$ as $E$. % TODO add ref fix up this line
This function is continuous on $\mathbb{T} \setminus H$, $u_H|_H = \infty$, and $\lim_{z \rightarrow w} u_H(z) = \infty$ for $w \in H$.
...
We now define
\[
	g_H(z) =
	\left \{
		\begin{array}{l l}
		u_H(z) + iv_H(z), & z \in \mathbb{D} \setminus H\\
		\infty, & \text{otherwise}
		\end{array}.
	\right .
\]
By our construction of $u_H$ we see the $\Re g > 1$, so it has a well defined square root. % TODO look better at this
Let's call it $h_H$ and define
\[
	q = \frac{h_{E_1}}{h_{E_0} + h_{E_1}}.
\]
Note that $|\arg h_H(z)| < \pi/4$ since if a $w \in \mathbb{C}$ had an argument outside of this range then its square would have and argument outside of the range $[-\pi/2, \pi/2]$ meaning $\Re w^2 < 0$.
Also, $q(z) = 0$ if and only if $h_{E_0} = \infty$, so $q$ is zero only on $E_0$, and $q(z) = 1$ if and only if $h_{E_1} = \infty$, so $q$ is one only on $E_1$.
We now want to show that $0 \leq \Re q \leq 1$.
We will let $z, w \in \mathbb{C}$, with $|\arg z|, |\arg w| < \pi/4$ and $\Re z, \Re w > 1$, and show that $0 < \Re z/(w + z) < 1$.

Firstly note that
\[
	\frac{z}{w + z}
	=
	\frac{1}{w/z + 1}
\]
so 
\[
	\arg \frac{z}{w + z} = -\arg \left ( \frac{w}{z} + 1 \right )
\]
and
\[
	|\arg w/z| = |\arg w - \arg z| \leq |\arg w| + |\arg z| < \pi/4 + \pi/4 = \pi/2.
\]
So $w/z$ is in the right halfplane and, since $\arctan(y/x)$ is decreasing in $x$ for positive $y$, we get
\[
	\arg \frac{z}{w + z} = -\arg \left ( \frac{w}{z} + 1 \right ) < -\arg  \frac{w}{z}
\]
and thus
\[
	\left |\arg \frac{z}{w + z} \right | < \left |\arg \frac{w}{z} \right | < \pi/2.
\]
We know $\arctan(y/x)$ is decreasing in $x$ for positive $y$ because
\[
	\frac{d}{dx}\arctan \left (\frac{y}{x} \right ) = -\frac{y}{x^2 + y^2} < 0.
\]
So $0 < \Re z/(w + z)$.
Note that $0 < \Re z/(w + z) \implies 0 > \Re -w/(w + z)$ due to $z$ and $w$ being constrained in the same manner.
So
\begin{align*}
	0
	&> \Re \frac{-w}{w + z}\\
	&= \Re \frac{z - (z + w)}{w + z}\\
	&= \Re \left ( \frac{z}{w + z} - 1 \right )\\
	&= \Re \frac{z}{w + z} - 1\\
	\implies 1 &> \Re \frac{z}{z + w}.
\end{align*}
So we have shown that
\[
	0 < \Re \frac{z}{z + w} < 1.
\]

So we have constructed a function $q$ that maps $\overline{D}$ to the ribbon $\{z;\ 0 \leq \Re z \leq 1\}$.
We then let $\Phi$ be the conformal mapping from the ribbon $\{z;\ 0 \leq \Re z \leq 1\}$ to $\{z;\ 0 \leq \Re z \leq \Re \alpha\}$. % TODO Rudin calls these ribbons rectangles, but couldn't the imaginary part be unbounded?
We will also choose $\Phi$ such that $\Phi(0) = 0$ and $\Phi(1) = \alpha$.
We can then let $f = \Phi \circ q$ and conclude the proof.

\end{proof}
\begin{lemma}
If $f$ is a simple continuous function on $E$ that maps $E$ into a simply connected $T$, then there exists a $F \in \mathcal{A}$, such that $F = f$ on $E$ and $F$ maps $\overline{\mathbb{D}}$ into $T$.
\end{lemma}
\begin{proof}
TODO
\end{proof}
\begin{lemma}
If $f$ is a continue function on $E$ which maps $E$ into $S = \{z;\ |\max(\Re z, \Im z)| \leq 1\}$, then there exists a sequence $(f_n)_{n \in \mathbb{N}}$ of simple continuous function on $E$ such that
\[
	f(x) = \sum_{n \in \mathbb{N}} f_n(z),
\]
\[
	f_n(E) \subset 2^{-n}S.
\]
\end{lemma}
\begin{proof}
We will set $f_0 = 0$ and construct $f_n$ iteratively.
Assuming $f_0, f_1, ..., f_{n - 1}$ have been constructed such that
\[
	\lambda_{n - 1}(E) \subset 2^{1 - n}S
\]
with $\lambda_{n - 1} = f - \sum_{k = 0}^{n - 1}f_k$.
According to \ref{somelemma} we have can write $E$ as the union of disjoint closed sets $E_1, E_2, ..., E_p$ such that the oscillation of $\lambda_{n - 1}$ is less than $2^{-n}$ on each $E_k$. % TODO make this lemma and maybe rewrite in accordance
So we can define $Q_k \subset 2^{1 - n}S$ for $k = 1, 2, ..., p$ such that $Q_k = 2^{-n}S + a_k$ for some $a_k \in S$.
We can choose $c_k \in Q_k \cap 2^{-n}$ since $Q_k$ has side length $2^{-n}$ and is a subset of $2^{-n + 1}$ and both are closed.
We can now define
\[
	f_n(z) = c_k, \tag*{$z \in E_k,\ k = 1, 2, ..., p$.}
\]
If we then look at $\lambda_n = f - \sum_{k = 0}^nf_k = \lambda_{n - 1} - f_n$ we see that $\lambda_n(E) \subset 2^{-n}S$ due to the way we decomposed $E$ into $E_1, E_2, ..., E_p$ using the oscialltions of $\lambda_{n - 1}$.
This means we can continue the process.
\end{proof}
\begin{proof}[Proof of \ref{actually ref to Rudin-Carleson}] % TODO fix the ref
Let's first show the result for $T = S$.

Let $f_n$ be the functions from Lemma \ref{lemma-3-from-Rudin}. % TODO fix the ref
According to Lemma \ref{lemma-2-from-Rudin} we have functions $g_n \in \mathcal{A}$ which extend $f_n$ and map $\overline{\mathbb{D}}$ into $2^{-n}S$.
We then define
\[
	F = \sum_{n \in \mathbb{N}} g_n
\]
on $\overline{\mathbb{D}}$.
To show that $F$ is in $\mathcal{A}$ we have to show it is holomorphic on $\mathbb{D}$ and continuous on $\overline{\mathbb{D}}$.
Both of these things can be shown by demonstarting that the series converges uniformly.
Let $M_n = 2^{-n + 1}$ and note that $|g_n(z)| \leq \sqrt{2}\cdot 2^{-n} < 2^{-n + 1} = M_n$ and
\[
	\sum_{n \in \mathbb{N}} M_n
	=
	\sqrt{2}\sum_{n \in \mathbb{N}} 2^{-n} =
	\sqrt{2} < \infty
\]
so the Weierstrass M-test tells us that $F$ converges uniformly. % TODO proof this and add a ref
This in turn tells that $F$ is both holomorphic on $\mathbb{D}$ and continuous on $\overline{mathbb{D}}$ so it is in $\mathcal{A}$.
We also have that
\[
	\Re F = \sum_{n \in \mathbb{N}} \Re g_n \leq \sum_{n \in \mathbb{N}} 2^{-n} = 1.
\]
It can be shown in the same manner that $\Im F \leq 1$, so $F$ maps into $S$.
Lastly, for $z \in E$ we have that
\[
	F(z) = \sum_{n \in \mathbb{N}} g_n(z) = \sum_{n \in \mathbb{N}} f_n(z) = f(z)
\]
so $F$ is an extension of $f$.

To proof the result for a general $T$ we first let $\Phi$ a the conformal mapping from $S$ to $T$, its existence provided by the Riemann mapping theorem. % TODO add ref to RMT and add the theorem itself
We will also let $g = f \circ \Phi^{-1}$.
Note that it maps $E$ into $S$, so we can use what we showed above to find $G \mathcal{A}$ that extends $g$ and maps into $S$.
We finally set $F = G \circ \Phi$.
% TODO finish this. All the is left is to show that F is holomorphic (easy, it's the composition of two holomorphic functions) and continuous (a bit of a different tale)
\end{proof}
\begin{corollary}[Fatou]
Let $E$ be a closed subset of $\mathbb{T}$ of Lebesgue-measure $0$.
There exists a function $f \in \mathcal{A}$ that vanishes on $E$ and nowhere else.
\end{corollary}
\begin{proof}
It's clear from the theorem % TODO add ref
that there exists a function $f \in \mathcal{A}$ that vanishes on $E$.
TODO
\end{proof}




\chapter{F. and M. Riesz theorem}
In this section we will endeavour to show that the annihilating measures of $\mathcal{A}|_{\mathbb{T}}$ are absolutely continuous with respect to the Lebesgue measure.
We will show this to be a corollary of the F. and M. Riesz theorem, which we will prove in the manner of Rudin. %TODO add the reference to Rudin real and complex analysis.
To attain the main result of this section we need some lemmas and definitions. %TODO rewrite these two lines
To prove one of the lemmas we will also use the following two famous theorems:
\begin{definition}
Let $\mathcal{F}$ be a family of complex functions on a metric space $(X, d)$.

We say that the family is \emph{pointwise bounded} if for all $x \in X$ there exists a constant $M < \infty$ such that
\[
	|f(x)| < M,\text{ for all } f \in \mathcal{F}.
\]
Note that $M$ is dependent on $x$.

We say that the family is \emph{equicontinuous} if for all $\varepsilon > 0$ there exists a $\delta > 0$ such that
\[
	|f(x) - f(y)| < \varepsilon,\text{ for all } f \in \mathcal{F}\text{ and } x, y \in X\text{ such that } d(x, y) < \delta.
\]
Note here that $\delta$ is globally defined and only dependent on $\varepsilon$.
\end{definition}
\begin{theorem}[Bolzano-Weierstrass]
%                                                                                           v This has to be weak because of (M, M, M, M, M, M, ...)
Let $(a_n)_{n \in \mathbb{N}}$ be a sequence of numbers in $\mathbb{R}^n$, such that $|a_n| < M < \infty$, for all $k \in \mathbb{N}$.
There than exists and infinite $S \subset \mathbb{N}$ such that $(a_n)_{n \in S}$ is convergent.
\end{theorem}
\begin{proof}
Let's first assume that the sequence is in $\mathbb{R}$,
	that no element in it is repeated infinitely often (there is nothing to prove in that case),
	and that $a_n \in ]0, 1[$ for all $n \in \mathbb{N}$.
The last assumption can be done with out loss of generality by studying the sequence $((a_n + M)/(2M))_{n \in \mathbb{N}}$ instead.
We will obtain the subsequence by a diagonal process.
Let $S_0 = \mathbb{N}$, $S_0^- = \{n \in S_0;\ a_n < 1/2\}$, and $S_0^+ = \{n \in S_0;\ a_n > 1/2\}$.
We then set $S_1 = S_0^-$ if it is infinite, but $S_1 = S_0^+$ otherwise.
This gives us a subsequence $(a_n)_{n \in S_1}$ such that
\[
	\sup_{n \in S_1} a_n - \inf_{n \in S_1} a_n < 1/2.
\]
We can then repeat this to get a sequence of sets $(S_n)_{n \in \mathbb{N}}$ such that $S_0 \supset S_1 \supset S_2 \supset...$ and 
\[
	\sup_{n \in S_k} a_n - \inf_{n \in S_k} a_n < 2^{1 - k},
\]
for all $k \in \mathbb{N}$.
Specifically, if we have $S_k$ we set
\[
	U = m2^{-k},\ 
	L = (m + 1)2^{-k}
\]
$S_k^- = \{n \in S_k;\ a_n < (U + L)/2\}$, and $S_k^+ = \{n \in S_k;\ a_n > (U + L)/2\}$.
We now set $S_{k + 1} = S_k^-$ if it has infinitely many elements, otherwise we set $S_{k + 1} = S_k^+$.
We conclude our construction by setting
\[
	S = \bigcup_{n \in \mathbb{N}} r_n,
\]
where $r_n$ is the $n$-th smallest element of $S_n$.
This gives us the convergent sequence $(a_n)_{n \in S}$ with limit
\[
	\sum_{k = 0}^{\infty} \delta_k 2^{-k}
\]
where
\[
	\delta_k = \left \{
	\begin{array}{c c}
		0,\ & \text{if we chose } S_k^- \\
		1,\ & \text{if we chose } S_k^+
	\end{array}.
	\right .
\]

To show the result for $\mathbb{R}^n$ we can start by finding a subsequence such that the first coordinate is convergent.
We can then chose a subsequence thereof such that the second coordinate is also convergent.
Now the first two coordinates are convergent.
If we do this $n - 2$ more times we get a desired subsequence.
% TODO What does (a_n)_{n \in S} mean????
\end{proof}
\begin{remark}
The theorem above clearly holds for sequences in $\mathbb{C}$ as well.
\end{remark}
\begin{theorem}[Ascoli-Arzela]
Let $\mathcal{F}$ be a pointwise bounded equicontinuous collection of complex functions on a metric space $(X, d)$, and $X$ contains a countable dense subset.  %TODO change to 'X is separable'
Then every sequence in $\mathcal{F}$ contains a subsequence that converges uniformly on every compact subsets of $X$.
\end{theorem}
\begin{proof}
Let $E$ be a countable dense subset of $X$,
	$(f_n)_{n \in \mathbb{N}}$ be a series in $\mathcal{F}$,
	and $x_1, x_2, ...$ be an enumeration of $E$.
We will prove the theorem in two steps.
The first step is finding a subsequence of $(f_n)_{n \in \mathbb{N}}$ that's pointwise convergent on $E$ using the point wise boundedness along with Bolzano-Weierstrass. %TODO add ref
The second step is using the equicontinuity to show that this gives us uniform continuity on compact subsets.

Let's first set $S_0 = \mathbb{N}$.
Pointwise boundedness gives us that the sequence $(f_n(x_1))_{n \in S_0}$ has a convergent subsequence.
Let $S_1$ index that subsequence.
We can use this process to generate sets $S_0 \supset S_1 \supset ...$ such that $(f_n(x_k))_{n \in S_k}$ is convergent.
We then set
\[
	S = \bigcup_{k \in \mathbb{N}} r_k
\]
where $r_n$ is the $k$-th smallest element of $S_k$.
We now have concluded the first step of the proof.

We will now assume the $(f_n)_{n \in \mathbb{N}}$ is pointwise convergent on $E$,
	let $K$ be a compact subset of $X$, and
	$\varepsilon > 0$.
Equicontinuity gives us a $\delta > 0$ such that $d(x, y) < \delta$ implies that $|f_n(x) - f_n(y)| < \varepsilon /3$, for all $n$.
Let's now cover $K$ with $m$ balls of radius $\delta /2$ and call the $k$-th ball $B_k$.
We can now set $p_k$ as a point in $B_k$.
This point exists because $E$ is dense in $X$.
Pointwise convergence on $E$ let's us chose an $N$ such that $|f_{n_1}(p_k) - f_{n_2}(p_k)| < \varepsilon /3$ for $k = 1, 2, ..., m$ and all $n_1, n_2 > N$.
Let's conclude by setting $x \in K$.
Then there is a $k$ such that $x \in B_k$ and thus $d(x, p_k) < \delta$.
The choice of $\delta$ and $N$ then gives us that
\begin{align*}
	|f_{n_1}(x) - f_{n_2}(x)|
	& \leq |f_{n_1}(x) - f_{n_1}(p_k)| + |f_{n_1}(p_k) - f_{n_2}(p_k)| + |f_{n_2}(p_k) - f_{n_2}(x)|\\
	& < \varepsilon /3 + \varepsilon /3 + \varepsilon /3 \\
	& = \varepsilon.
\end{align*}
\end{proof}
\begin{definition}
% TODO finish these
Poisson kernel, Poisson integral, Poisson integral of a measure.
\end{definition}
\begin{lemma}
\label{FMRlemma1}
Let $\mu$ be a complex Borel measure, and $u = P[d\mu]$.
Then
\[
	\|u_r\|_1 \leq \|\mu\|.
\]
\end{lemma}
\begin{proof}
% TODO reword this part
First, we need to see that, if $n \neq 0$
\[
	in\int_{-\pi}^{\pi} e^{int}\ dt
	= (e^{in\pi} - e^{-in\pi})
	= (e^{in\pi} - e^{-i(2n\pi - n\pi)})
	= (e^{in\pi} - e^{in\pi})
	= 0,
\]
so
\begin{align*}
	\int_{-\pi}^{\pi} P_r(t)\ dt
	&= \int_{-\pi}^{\pi} \sum_{n \in \mathbb{Z}} r^{|n|}e^{int}\ dt\\
	&= \sum_{n \in \mathbb{Z}} \int_{-\pi}^{\pi} r^{|n|}e^{int}\ dt\\
	&= \int_{-\pi}^{\pi} \ dt\\
	&= 2\pi.
\end{align*}
Fubini let's us
\begin{align*}
	\|u\|_1
	&= \frac{1}{2\pi} \int_{-\pi}^{\pi} |u(re^{i\theta})| d\theta\\
	&= \frac{1}{2\pi} \int_{-\pi}^{\pi} \left | \int_{\mathbb{T}} P(re^{i\theta}, e^{it})\ d\mu(e^{it}) \right | d\theta\\
	&\leq \frac{1}{2\pi} \int_{-\pi}^{\pi} \int_{\mathbb{T}} P(re^{i\theta}, e^{it})\ d|\mu(e^{it})| d\theta\\
	&= \int_{\mathbb{T}} \frac{1}{2\pi} \int_{-\pi}^{\pi} P(re^{i\theta}, e^{it})\ d\theta d|\mu(e^{it})|\\
	&= \int_{\mathbb{T}}\ d|\mu(e^{it})|\\
	&= |\mu|(\mathbb{T})\\
	&= \|\mu\|.
\end{align*}
\end{proof}
\begin{lemma}
\label{FMRlemma2}
Let $f \in H^1$.
Then there exists a $g \in L^1(\mathbb{T})$ such that $f = P[g]$.
\end{lemma}
\begin{proof}
% TODO
\end{proof}
\begin{lemma}
%11.30(a) in Rudin
\label{FMRlemma3}
Let $u$ be harmonic in $\mathbb{D}$ and
\[
	\sup_{0 < r < 1} \|u_r\|_1 = M < \infty. %TODO define u_r
\]
Then there exists a unique complex Borel measure $\mu$ on $\mathbb{T}$ such that $u = P[d\mu]$.
\end{lemma}
We will need the following lemma in the proof of \ref{FMRlemma3}:
\begin{lemma}
\label{FMRlemma31}
Let $X$ be a separable Banach space,
	$(\Gamma_n)_{n \in \mathbb{N}}$ be a sequence of linear functionals on $X$,
	and $\sup_n \|\Gamma_n\| = M < \infty$.
Then there exists a subsequence $\{\Gamma_{n_i}\}$ such that the limit 
\[
	\Gamma x = \lim_{k \rightarrow \infty} \Gamma_{n_k}\ x
\]
exists for every $x \in X$.
We also have that $\Gamma$ is linear and $\|\Gamma\| \leq M$.
\end{lemma}
\begin{proof}
We have that $|\Gamma_n\ x| \leq M \|x\|$ and
\begin{align*}
	|\Gamma_n\ x - \Gamma_n\ y|
	&= |\Gamma_n (x - y)|\\
	&\leq M\|x - y\|.
\end{align*}
The first inequality gives us pointwise boundedness and the second gives us equicontinuity.
Now, since singletons are compact, Ascoli-Arzela gives us a subsequence, let's index it by $S$, such that $(\Gamma_n x)_{n \in S}$ is convergent for all $x \in X$. % TODO add ref to Ascoli-Arzela
Let's now define $\Gamma$ by
\[
	\Gamma(x) = \lim_{k \in S} \Gamma_k\ x,
\]
see the
\begin{align*}
	\Gamma(x) + \Gamma(y)
	&= \lim_{k \in S} \Gamma_k\ x + \lim_{k \in S} \Gamma_k\ y\\
	&= \lim_{k \in S} \left ( \Gamma_k\ x + \Gamma_k\ y \right )\\
	&= \lim_{k \in S} \Gamma_k(x + y)\\
	&= \Gamma(x + y),
\end{align*}
where the third equality holds because addition is continuous, and $a\Gamma(x) = \Gamma(ax)$ obviously holds.
So $\Gamma$ is linear.
Lastly
\begin{align*}
	\|\Gamma\|
	&= \sup \{|\Gamma x|;\ \|x\| \leq 1\}\\
	&= \sup \left \{ \left |\lim_{n \in S} \Gamma_n x \right|;\ \|x\| \leq 1 \right \}\\
	&\leq \sup \{ M;\ \|x\| \leq 1 \}\\
	&= M.
\end{align*}
\end{proof}
\begin{proof}[Proof of \ref{FMRlemma3}]
% TODO fix the Riesz rep thm ref. It's 6.19 in Rudin.
Let $\Gamma_r$, for $r \in [0, 1[$,  be linear functionals on $C(\mathbb{T})$ defined by
\[
	\Gamma_r g = \int_{\mathbb{T}} gu_r d\sigma.
\]
If $\|g\| \leq 1$ is assummed we get that
\[
\Gamma_r\ g = \int_{\mathbb{T}} gu_r\ d\sigma \leq \int_{\mathbb{T}} u_r\ d\sigma = \|u_r\|_1 \leq M.
\]
so
\[
	\|\Gamma_r\| \leq M.
\]
By the above lemma % TODO add ref
and the Riezs representation theorem % TODO add ref
we get a measure $\mu$ on $\mathbb{T}$ with $\|\mu\| \leq M$, and a sequence $(r_n)_{n \in \mathbb{N}}$ on $[0, 1[$ with limit $1$, such that
\begin{equation}
	\label{eq:1}
	\lim_{n \rightarrow \infty} \int_{\mathbb{T}} gu_{r_n}\ d\sigma = \int_{\mathbb{T}}g\ d\mu
\end{equation}
for all $g \in C(\mathbb{T})$.
Let's now define functions $h_k$ on $\overline{\mathbb{B}}$ by $h_k(z) = u(r_kz)$.
We get that, since $u$ is harmonic on $r\mathbb{B}$ for $r \in ]0, 1[$, the functions $h_k$ are harmonic on $\mathbb{B}$ and continuous on $\overline{\mathbb{B}}$.
So each of them can be represented by the Poisson integral of their restriction to $\mathbb{T}$, according to Ramsford. % TODO add ref to Ramsford, solution of the Dirichlet problem
Note that $h_k(e^{it}) = u_{r_k}(e^{it})$, so
% TODO not use h. instead just looking at u_{r_n} should work.
\begin{align*}
	u(z)
	&= \lim_{n \rightarrow \infty} u(r_nz)\\
	&= \lim_{n \rightarrow \infty} h_n(z)\\
	&= \lim_{n \rightarrow \infty} \int_{\mathbb{T}}P(z, e^{it}) h_n(e^{it})\ d\sigma(e^{it})\\
	&= \lim_{n \rightarrow \infty} \int_{\mathbb{T}}P(z, e^{it}) u_{r_n}(e^{it})\ d\sigma(e^{it})\\
	&= \int_{\mathbb{T}}P(z, e^{it})\ d\mu(e^{it})\\
	&= P[d\mu](z),
\end{align*}
where the fifth equlity is achived by putting $g = P(z, e^{it})$ into \ref{eq:1}.
This concludes the proof of excistence.
% TODO show uniqueness

Let's assume that $P[d\mu] = 0$, and let $f \in C(\mathbb{T})$, $u = P[f]$ and $v = P[d\mu]$.
We firstly have the symmetry
\[
	P(re^{i\theta}, e^{it})
	=
	P(re^{it}, e^{i\theta}).
\]
This symmetry is due to
\[
	|e^{it} - re^{i\theta}|
	=
	|1 - re^{i(\theta - t)}|
	=
	|1 - re^{i(t - \theta)}|
	=
	|e^{i\theta} - re^{it}|,
\]
which is geometrically intuitive.
The first and last equalities hold because the euclidian metric is rotationally invariant, and the second euality holds becuase the distance from $z$ to a real number $a$ is the same distance from $\overline{z}$ to $a$.
We now obtain
\begin{align*}
\int_{\mathbb{T}} u_r d\mu
&= \int_{\mathbb{T}} \int_{-\pi}^{\pi} P(re^{i\theta}, e^{it}) f(e^{i\theta})\ d\theta d\mu(e^{it})\\
&= \int_{-\pi}^{\pi} f(e^{i\theta}) \int_{\mathbb{T}} P(re^{it}, e^{i\theta})\ d\mu(e^{it}) d\theta\\
&= \int_{-\pi}^{\pi} f(e^{i\theta}) v_r\ d\theta\\
&= \int_{\mathbb{T}} fv_r\ d\sigma.
\end{align*}
If we let $r \rightarrow 1$ we get
\[
	\int_{\mathbb{T}}f\ d\mu = 0.
\]
This holds for all $f \in C(\mathbb{T})$, so the measure $\mu$ represents zero in the dual of $C(\mathbb{T})$.
The Riesz representation theorem then tells us that $|\mu|(\mathbb{T}) = 0$, so $\mu = 0$.

Now let $\lambda$ and $\nu$ be measures on $\mathbb{T}$ such that $P[d\lambda] = P[d\nu]$.
We have that $P[d(\lambda - \nu)] = 0$, so, as shown above $\lambda - \nu = 0$.
Moreover $\lambda = \nu$, which concludes the proof of uniquness.
\end{proof}

\begin{theorem}[F. and M. Riesz theorem]
If $\mu$ is a complex Borel measure on $\mathbb{T}$ and
\[
	\int e^{-int} d\mu = 0
\]
for $n = -1, -2, ...$, then $\mu \lll m$.
\end{theorem}
\begin{proof}
Let $f = P[d\mu]$.
If we set $z = re^{i\theta}$ we get that
\[
	P(z, e^{it})
		= P_r(\theta - t)
		= \sum_{n \in \mathbb{Z}} r^{|n|}e^{in(\theta - t)}
		= \sum_{n \in \mathbb{Z}} r^{|n|}e^{in\theta}e^{-int}.
\]
We can use the assumption of the theorem to write $f$ as a power series by
\begin{align*}
% TODO Fubini?
	f(z) 
		&= \int_{\mathbb{T}} P(z, e^{it}) d\mu(e^{it}) \\
		&= \int_{\mathbb{T}} \sum_{n \in \mathbb{Z}} r^{|n|}e^{in\theta}e^{-int} d\mu(e^{it}) \\
		&= \sum_{n \in \mathbb{Z}} r^{|n|}e^{in\theta} \int_{\mathbb{T}} e^{-int} d\mu(e^{it}) \\
		&= \sum_{n = 0}^{\infty} r^n e^{in\theta} \int_{\mathbb{T}} e^{-int} d\mu(e^{it}) \\
		&= \sum_{n = 0}^{\infty} \hat{\mu}_n z^n,
\end{align*}
where $\hat{\mu}_n$ is the $n$-th Fourier coefficient of $\mu$.
This along with \ref{FMRlemma1} gives us that $f \in H^1$. % TODO make this more clear
We can now define a $g \in H^1$, by \ref{FMRlemma2}, such that $f = P[g]$.
It follows from \ref{FMRlemma3} that $d\mu = f d\sigma$. %TODO define \sigma as Lebesgue on T scaled to 1.
TODO
\end{proof}
\begin{corollary}
Let $A$ be the closed subspace of $C(\mathbb{T})$ that consists of all functions that are restriction from $\mathcal{A}$. % TODO makes this def global and maybe call it something else
All measures in $A^{\bot}$ are absolutely continuous with regards to the Lebesgue-measure on $\mathbb{T}$. % TODO define all
\end{corollary}
\begin{proof}
Let $\mu \in A^{\bot}$. By definition we have that
\[
	\int fd\mu = 0.
\]
Now since $t \mapsto e^{-int}$ is entire for $n = -1, -2, ...$ we have that their restriction to $\mathbb{T}$ are in $A$.
Thus,
\[
	\int e^{-int} d\mu = 0
\]
for all $n = -1, -2, ...$ and $\mu \lll m$.
\end{proof}



\chapter{A generalization of the Rudin-Carleson theorem}
This borrows from Bishop. % TODO add reference
\begin{theorem}[General Rudin-Carleson theorem]
Let $X$ be a compact Hausdorff space,
	$V = (C(X), \| \cdot \|_{\infty})$,
	$B$ be a closed subspace of $C(X)$,
	$B^{\bot}$ be the annihilating measures for $B$,
	$S$ be a closed subset of $X$,
	and $f$ be a continues function on $S$.
If $\int_S f d\mu = 0$ holds for all $\mu \in B^{\bot}$ then there exists a function $F \in B$ such that $F = f$ on $S$.
\end{theorem}
\begin{proof}
Since $f$ is continuous and $S$ is a closed subset of a compact set, and therefore also compact, $f$ is bounded.
So we can, with out loss of generality, assume that $|f| < r < 1$ on $S$.
Let $U_r$ be the subset of $B$ defined by $U_r = \{g;\ \|g\| < r\}$ and $\phi$ be the mapping from $B$ to $C(S)$ that sends a member of $B$ to its restriction on $S$.
It suffices to show that $f \in \phi(U_r)$.
Let's first show that $f \in \overline{\phi(U_r)} =: V_r$, by assuming otherwise, and showing it leads to a contradiction.

We now assume $f \not \in V_r$.
By Hahn-Banach % TODO ref
we can define a bounded linear functional $\alpha$, such that $\alpha(f) > 1$ and $|\alpha(h)| < 1$, for $h \in V_r$.
We can then define a measure $\mu_1$ by the Riesz-representation theorem % TODO ref
that fulfills
\[
	\alpha(g) = \int g d\mu_1
\]
for all $g \in C(S)$.
We will refer to the associated functional on $B$ by $\beta(g) = \phi(\alpha(g))$.
Since $\phi(g) \in V_r$ for all $g \in U_r$ we have that
\[
	\beta(g) = \alpha(\phi(g)) < 1,
\]
for all $g \in U_r$,
	due to the construction of $\alpha$.
From this we get
\begin{align*}
	\| \beta \|
	&= \sup \{ |\beta(g)|;\ |g| < 1 \}\\
	&= \sup \{ (1/r)|\beta(g)|;\ |g| < r \}\\
	&\leq \sup \{ (1/r);\ |g| < r \}\\
	&= 1/r.
\end{align*}
Let's denote the Riesz representation of $\beta$ by $\mu_2$, set $\mu = \mu_1 - \mu_2$ and see that $\mu \in B^{\bot}$.
But
\[
	0 = \left | \int_S f d\mu \right | 
		\geq \int_S f d\mu_1 - r\|\mu_2\| 
		\geq \int_S f d\mu_1 - r\frac{1}{r}
		> 1 - r \frac{1}{r} = 0,
\]
where the first equality is the assumption in the theorem.
This is the contradiction that gives that $f \in V_r$.
We can now take a $F_1$ in $U_r$, and therefore also in $B$ such that $|f - F_1| < \lambda/2$ on $S$, with $\lambda := 1 - r$.
Remember that $F_1 \ in U_r$ implies that $\|F_1\| < r$.
Now let $f_1 = f - F_1$ and use the same method as above to obtain an $F_2$ such that $\|F_2\| < \lambda/2$ and $|f - F_2| < \lambda/4$ on $S$.
Iterating this process yields a series $(F_n)_{n \in \mathbb{N}}$ from $B$ that fulfill $\|F_n\| < 2^{1 - n}\lambda$ for $n > 1$ and
\[
	\left | f - \sum_{k = 1}^n F_k \right | < 2^{-n}\lambda
\]
on $S$ for $n > 1$.
We finally let 
\[
F = \sum_{k = 1}^{\infty} F_k.
\]
Now $F \in B$,
\[
	\|F\| \leq \|F_1\| + \|F - F_1\| = r + \sum_{k = 2}^{\infty}2^{1 - n}\lambda = r + \lambda = 1,
\]
and $F = f$ on $S$. % TODO bæta við matinu, svo línan að ofan meiki sens
\end{proof}
\begin{corollary}
Let $X$ be a compact Hausdorff space,
	$V = (C(X), \| \cdot \|_{\infty})$,
	$B$ be a closed subspace of $C(X)$,
	$B^{\bot}$ be the annihilating measures for $B$,
	$S$ be a closed subset of $X$,
	and $f$ be a continues function on $S$.
If $S$ is $B^{\bot}$-null % TODO define this
then there exists a function $F \in B$ such that $F = f$ on $S$.
\end{corollary}
\begin{proof}
If $S$ is $B^{\bot}$-null we have that $\int_S f d\mu = 0$ for all $\mu \in B^{\bot}$. % TODO add ref to above
\end{proof}
\begin{remark}
The corollary is the version of the theorem from Bishop. % TODO add ref here
Note also that if we set $X = \mathbb{T}$ and $B = \mathcal{A}$ we can use F. and M. Riesz % TODO add ref here
to prove the classical Rudin-Carleson theorem. % TODO add ref here
\end{remark}
It is of course worth noting an applications of where the corollary fails.  % TODO REF to theorem)
\begin{example}
Let $X = \mathbb{T}$,
	$B = \mathcal{A}$, % TODO define this
	$E$ be a closed $m$-null % TODO define this, an maybe change to $m_{\sigma}$
	subset of $\partial \mathbb{T}$ that is not dense in $E$,
	$F = \{ e^{i\theta};\ a \leq \theta \leq b \}$,
	and choose $a$ and such that $E$ and $F$ are disjoint and $a \neq b$.
The last assumption restricts us to $E$ that are not dense in the $\mathbb{E}$.
Since $a \neq b$ we obtain that $S := E \cup F$ does not fulfill the requirements of the Rudin-Carelson theorem nor the above corollary. % TODO add refs
Let's choose $f$ such that $f = 0$ on $F$, and $f$ is continues on $S$.
We now have for all $\mu \in \mathcal{A}^{\bot}$
\begin{align*}
	\left | \int_S f d\mu \right |
	&= \left |\int_E f d\mu + \int_F f d\mu \right |\\
	&\leq \left |\int_E f d\mu \right | + \left | \int_F f d\mu \right |\\
	&= 0 + \left | \int_F f d\mu \right |\\
	&= 0.
\end{align*}
The F. and M. Riesz theorem %TODO add ref here
tells us the since $E$ is $m$-null it is also $\mu$-null, which gives the third step.
The final step stems from the fact that $f$ vanishes on $F$.
We now see the $X$, $B$, and $f$ are all as in theorem % TODO add ref
so there exists a $F \in B$, such that $F = f$ on S.
\end{example}

\appendix
\renewcommand{\chaptername}{Appendix}
\chapter{Annað}

\end{document}
